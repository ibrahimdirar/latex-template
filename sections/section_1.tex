\section*{Section title}
\lipsum


Algorithm~\ref{algo:algo_example} is a good example of an algorithm in \LaTeX.
\begin{algorithm}
    \caption{Example caption: sum of all even numbers}
    \label{algo:algo_example}
    \begin{algorithmic}[1]
        \Require{$ \mathbf{x}  = x_1, x_2, \ldots, x_N$}
        \Ensure{$EvenSum$ (Sum of even numbers in $ \mathbf{x} $)}
        \Statex
        \Function{EvenSummation}{$\mathbf{x}$}
        \State {$EvenSum$ $\gets$ {$0$}}
        \State {$N$ $\gets$ {$length(\mathbf{x})$}}
        \For{$i \gets 1$ to $N$}
        \If{$ x_i\mod 2 == 0$}  \Comment check if a number is even?
        \State {$EvenSum$ $\gets$ {$EvenSum + x_i$}}
        \EndIf
        \EndFor
        \State \Return {$EvenSum$}
        \EndFunction
    \end{algorithmic}
\end{algorithm}

Code Listing~\ref{list:python_code_ex} is a good example of including a code snippet in a report. While using code snippets, take care of the following:
\begin{itemize}
    \item do not paste your entire code (implementation) or everything you have coded. Add code snippets only.
    \item The algorithm shown in Algorithm~\ref{algo:algo_example} is usually preferred over code snippets in a technical/scientific report.
    \item Make sure the entire code snippet or algorithm stays on a single page and does not overflow to another page(s).
\end{itemize}

Here are three examples of code snippets for three different languages (Python, Java, and CPP) illustrated in Listings~\ref{list:python_code_ex}, \ref{list:java_code_ex}, and \ref{list:cpp_code_ex} respectively.

\begin{lstlisting}[language=Python, caption={Code snippet in \LaTeX ~and  this is a Python code example}, label=list:python_code_ex]
import numpy as np

x  = [0, 1, 2, 3, 4, 5] # assign values to an array
evenSum = evenSummation(x) # call a function

def evenSummation(x):
    evenSum = 0
    n = len(x)
    for i in range(n):
        if np.mod(x[i],2) == 0: # check if a number is even?
            evenSum = evenSum + x[i]
    return evenSum
\end{lstlisting}

Here we used  the ``\textbackslash clearpage'' command and forced-out the second listing example onto the next page.
\clearpage  %
\begin{lstlisting}[language=Java, caption={Code snippet in \LaTeX ~and  this is a Java code example}, label=list:java_code_ex]
public class EvenSum{ 
    public static int evenSummation(int[] x){
        int evenSum = 0;
        int n = x.length;
        for(int i = 0; i < n; i++){
            if(x[i]%2 == 0){ // check if a number is even?
                evenSum = evenSum + x[i];
            }
        }
        return evenSum;     
    }
    public static void main(String[] args){ 
        int[] x  = {0, 1, 2, 3, 4, 5}; // assign values to an array
        int evenSum = evenSummation(x);
        System.out.println(evenSum);
    } 
} 
\end{lstlisting}


\begin{lstlisting}[language=C, caption={Code snippet in \LaTeX ~and  this is a C/C++ code example}, label=list:cpp_code_ex]
int evenSummation(int x[]){
    int evenSum = 0;
    int n = sizeof(x);
    for(int i = 0; i < n; i++){
        if(x[i]%2 == 0){ // check if a number is even?
            evenSum = evenSum + x[i];
    	}
    }
    return evenSum;     
}

int main(){
    int x[]  = {0, 1, 2, 3, 4, 5}; // assign values to an array
    int evenSum = evenSummation(x);
    cout<<evenSum;
    return 0;
}
\end{lstlisting}